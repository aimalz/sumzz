\documentclass[11pt]{amsart}
\usepackage{geometry}                % See geometry.pdf to learn the layout options. There are lots.
\geometry{letterpaper}                   % ... or a4paper or a5paper or ... 
%\geometry{landscape}                % Activate for for rotated page geometry
%\usepackage[parfill]{parskip}    % Activate to begin paragraphs with an empty line rather than an indent
\usepackage{graphicx}
\usepackage{amssymb}
\usepackage{epstopdf}
\DeclareGraphicsRule{.tif}{png}{.png}{`convert #1 `dirname #1`/`basename #1 .tif`.png}

\title{$p(z)$}
%\author{}
\date{\today}                                           % Activate to display a given date or no date

\begin{document}
\maketitle

These are a few notes on redshift PDF summation.

\section{Definitions}

Some definitions so I don't confuse them later
\begin{itemize}
\item $a$: a vector of parameters describing the true, noiseless observable properties of a galaxy; if you know $a$, you know everything there is to know about the galaxy, including its redshift
\item $z$: true redshift of a galaxy 
\item $d$: the data vector of a galaxy in some photometric survey; includes photometric measurements and their uncertainties
\end{itemize}
All of these come with a subscript $i$ for galaxy $i$ in the sky, and there is a set of all possible $a$ defined as $A$. 

In a Bayesian hierarchical scheme, there is a vector of hyperparameters $h\in H$ that determines $p(a)$ and because we don't know the true $h$ the two always appear together as $p(a,h)$. For now, assume we know the true $H$.
\section{Individual galaxy $p(z_i)$}

Let's look at a single galaxy $i$ and estimate its redshift based on photometric measurements $d_i$ in the survey, which is what, for instance, BPZ intends to do, i.e. calculate $p(z_i|d_i)$. The following is a simple equality:
\begin{equation}
p(z_i|d_i, H)=\int_{A} p(z_i|a) p(a|d_i, H) \; da = \int_{A} p(z_i|a) p(a, H) p(d_i|a) / p(d_i) \; da
\end{equation}

Of these
\begin{itemize}
\item $p(z_i|a)$ is simply $\delta(z_i-z(a))$ since we said $z$ was fully determined by $a$ 
\item $p(a, H)$ only depends on the Universe (i.e., the value that the hyperparameters happen to take)
\item $p(d_i|a)$ is very simple in the limit of noiseless $d_i$ (a multi-dimensional $\delta$ function, because here $d_i$ is equal to some subset of true observable properties $a_i$) and depends on the survey for noisy $d_i$
\item $p(d_i)$, the distribution of \emph{observed} parameters in our survey, depends a lot on the survey (because there is e.g. a spatially dependent magnitude limit), but it doesn't influence $p(z_i|d_i, H)$ because it doesn't depend on redshift and is fixed by normalization anyway ($1=\int_{\mathbb{R}} p(z_i|d_i)\; dz_i$).
\end{itemize}

The following statements about $p(z_i|d_i, H)$ hold:
\begin{itemize}
\item $p(z_i|d_i,  H)$ gets wider in $z_i$ if our survey gets worse. It does so because $p(d_i|a_i)$ changes
\item If we had many equally looking galaxies $i$ with $d_i=d \; \forall i$ then $p(z_i|d_i, H)$ is the same for all $i$. The true redshifts of these galaxies are all different, and are actually distributed as $p(z_i|d_i, H)$.
\item The latter requires that the galaxies are not just equally looking but \emph{fully representative} of the set of galaxies that are equally looking. Galaxies that look like $d_i$ \emph{but are selected by some properties that are not part of $d_i$} (typically the case for spectroscopic surveys, where the success rate of redshift determination depends on whether or not a galaxy has a bright emission line), you'll get a different $p(z)$ (see Daniel's draft).
\end{itemize}

\section{Ensemble $p_D(z)$}

Consider, for a change, the redshift distribution of an ensemble of galaxies. Assume that we have selected these galaxies by observational properties, e.g. we take all galaxies whose $d_i\in D$, where $D$ is some interval of outcomes of our photometric measurements. Call the true redshift distribution of this set of galaxies $p_D(z|H)$. The true redshift distribution is what you would get if you measure the true redshifts of all galaxies with $d\in D$ and bin up the results.

The following is an equality
\begin{equation}
p_D(z|H)=p(z|d\in D, H)=\int_{D}p(z_i|d_i, H) p(d_i|H)/p(d\in D|H) \; d d_i 
\end{equation}
The $p(d|H)/p(d\in D|H)$ is just to keep things normalized. It's the PDF of $d$ for galaxies in $D$ given hyperparameters $H$. In the limit of many galaxies $i=1,...,N$ that sample this PDF, 
\begin{equation}
\int_{D}p(z_i|d_i,H) p(d_i|H)/p(d\in D|H) \; d d_i \rightarrow \frac{1}{N} \sum_{i=1}^N p(z_i|d_i,H) \; ,
\end{equation}
so it's fine to sum the individual PDFs.

\section{If you are uncertain about $H$}

So far, we assumed that we know $p(a)$ exactly, or equivalently that we know some vector of hyperparameters $H$ exactly that determines $p(a|H)$. This is the endgame of galaxy evolution. What if we don't know $H$, we only know some $p(a|h)$ and $p(h)$? Then eqn. (1) becomes

\begin{equation}
p(z_i|d_i)=\int_{A}\int_{H} p(z_i|a) p(a|d_i,h) p(h) \; da\; dh = \int_{A}\int_{H} p(z_i|a) p(a|h) p(h) p(d_i|a,h) / p(d_i) \; da \; dh \; .
\end{equation}

Because $p(a,h)=p(a|h)p(h)$ and $p(d_i|a,h)$ depend on $h$ this PDF will actually be broader than the distribution of $z$ that we would get if we binned up the true redshifts of all galaxies that look like $d_i$. Likewise, the stacked PDF will be broader than the actual PDF on an ensemble of galaxies selected by $d\in D$. PDF stacking is not an operation that gets you to the ensemble PDF in a scheme with uncertainty in the hyperparameters. This means for example
\begin{itemize}
\item stacking PDFs estimated with different codes will give you the wrong PDF (both for a single galaxy and for an ensemble) because that implicitly includes uncertainty on the hyperparameters (chosen by the different codes)
\item A Bayesian hierarchical scheme is more complicated than the equation above because it uses the data to constrain the hyperparameters. But as long as they're not constrained completely, the stacked $p(z)$ of individual galaxies estimated from a Bayesian hierarchical scheme is not an unbiased estimate of their true PDF - it is wider than that. 
\item Actually, that is already true for the PDF of an individual galaxy. If you take spectra of many equal-looking galaxies then their distribution will be narrower than the above $p(z_i|d_i)$. That is because these spectra actually constrain $h$ (in the limit for infinitely many spectra, they'll fully constrain $h$ up to changes that are degenerate under the redshift distribution of galaxies that look like $d_i$).
\item In the previous sections, we had never assumed that the $H$ was correct, just that we weren't uncertain about it. So the $p(z_i|d_i,H)$ and $p_D(z_i|H)$ also work with little $h$ instead of $H$.
\end{itemize}



%\section{}
%\subsection{}



\end{document}  
