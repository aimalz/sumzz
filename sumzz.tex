\documentclass[11pt]{amsart}
\usepackage{geometry}                % See geometry.pdf to learn the layout options. There are lots.
\geometry{letterpaper}                   % ... or a4paper or a5paper or ...
%\geometry{landscape}                % Activate for for rotated page geometry
%\usepackage[parfill]{parskip}    % Activate to begin paragraphs with an empty line rather than an indent
\usepackage{graphicx}
\usepackage{amssymb}
\usepackage{epstopdf}
\usepackage{natbib}
\DeclareGraphicsRule{.tif}{png}{.png}{`convert #1 `dirname #1`/`basename #1 .tif`.png}


\title{How should we combine the posterior PDFs for the photometric redshifts of individual galaxies?}

\author{Daniel Gruen, Phil Marshall and Chris Davis}
\date{\today}


\begin{document}
\maketitle

\section{Introduction}

A common practice in observational cosmology is to sum the posterior
PDFs for the photometric redshifts of individual galaxies in order to
estimate the underlying redshift distribution of galaxies.  At first
glance this might seem like an odd thing to do: normally we take
products of probability distributions when combining information from
different sources. In these notes we explore the assumptions that are
being made when summing posteriors, and investigate ways to cope with
these assumptions.

\section{Definitions}

First, some definitions:
\begin{itemize}
\item Let $a$ be a vector of parameters describing the true, noiseless
observable properties of a galaxy; if you know $a$, you know everything
there is to know about the galaxy, including its redshift, which can therefore be written as $z(a)$.
\item Let $z$ be the true redshift of a galaxy.
\item Let $d$ be the data vector of a galaxy in some photometric survey;
it includes photometric measurements and their uncertainties.
\end{itemize}
All of these quatities come with a subscript $i$ for galaxy $i$ in the sky.
%, and we denote the set of all possible $a$ as $A$.

We imagine that there is a ``true model,'' with some unknown
parametrization, that can be thought of as having generated the data we
have. The goal in inference is to learn this model and its parameters.
However, in practice we do not have access to $T$, and so have to be
satisfied with assuming a model $M$ for any particular galaxy's redshift
and type, or better, a hierarchical model $H$ that describes the whole
population of galaxies. The parameters of $M$ include $a$, and also any
latent variables~$m$.  In a Bayesian hierarchical scheme, the redshift
distribution of galaxies is modeled explicitly, via assumptions $H$ and
a vector of hyperparameters $h$ that determines $P(z,a|h,H)$.

% , and because we don't know the true $h$ the two always appear
% together as $P(a,h)$. For now, assume we know the true $H$.

\section{Individual galaxy posteriors $P(z_i|d_i,M)$}

Let's look at a single galaxy $i$ and infer its redshift based on photometric measurements $d_i$ in the survey, which is what, for instance, BPZ intends to do. This inference involves some assumptions, which we denote by $M$, and so results in a characterisation of $P(z_i|d_i,M)$. The following are two simple equalities, the second of which uses Bayes' theorem:
\begin{equation}
P(z_i|d_i, M) = \int P(z_i|a) P(a|d_i, M) \; da = \int P(z_i|a) P(a|M) P(d_i|a,M) / P(d_i|M) \; da
\end{equation}

Of these,
\begin{itemize}
\item $P(z_i|a)$ is simply $\delta(z_i-z(a))$ since we said $z$ was fully determined by $a$.
\item $P(a|M)$ depends on the assumptions that we make when measuring photometric redshifts (such as the magnitude distribution, or the abundance of various galaxy types, or the template spectra used).
\item $P(d_i|a,M)$ is very simple in the limit of noiseless $d_i$: it is  a multi-dimensional $\delta$ function, because in our notation $d_i$ is equal to some subset of true observable properties $a_i$. In this limit, $P(d_i|a)$ independent of $M$. It depends on the survey, and hence $M$, for noisy $d_i$.
\item $P(d_i|M)$, the distribution of \emph{observed} parameters in our survey given our assumptions, depends a lot on the survey (because there is e.g. a spatially dependent magnitude limit), but it doesn't influence $P(z_i|d_i, M)$ because it doesn't depend on redshift and is fixed by normalization anyway ($1=\int_{\mathbb{R}} P(z_i|d_i)\; dz_i$), in which context it is sometimes known as the ``Evidence,'' ``Fully Marginalized Likelihood,'' or ``FML.''
\end{itemize}


\section{Working in the ``true model''}

Suppose we had a way of computing the posterior PDF for a galaxy redshift that (somehow) assumed the ``true model,'' $T$.

In this case, the following statements about $P(z_i|d_i, T)$ hold:
\begin{itemize}
\item The individual galaxy redshift posterior PDF $P(z_i|d_i, T)$ still gets wider in $z_i$ if our survey gets worse. It does so because $P(d_i|a_i,T)$ changes.
\item If we had many apparently identical galaxies $i$ with $d_i=d \; \forall i$ then $P(z_i|d_i, T)$ would be the same for all $i$. The true redshifts of these galaxies could all be different, but these redshifts would actually be distributed as $P(z_i|d_i, T)$.
\item For the latter statement to be true, we require that the galaxies are not just identical in their observed data $d$, but also \emph{fully representative} of the set of galaxies that look identical. Galaxies that look like $d_i$ {\it but are selected by some properties that are not part of $d_i$} would follow $P(z_i|d_i, S)$, where $S$ denotes some different selection than drawing faithfully from $P(d_i|T)$.
% In this case, you'll get a different $P(z)$ in the end (see Daniel's draft).
This is typically the case for spectroscopic surveys, where the success rate of redshift determination depends on whether or not a galaxy has a bright emission line.
\end{itemize}

Now let's consider the redshift distribution of an ensemble of galaxies.
Assume that we have selected these galaxies by observational properties,
e.g. we take all galaxies whose $d_i\in D$, where $D$ is some interval
of outcomes of our photometric measurements. Call the true redshift
distribution of this set of galaxies $P_D(z|T)$. If you measured the
true redshifts of all galaxies with $d\in D$, for a complete set of bins
$D$, and plotted up the results, you would get a histogram-like
piece-wise constant approximation to the global true redshift
distribution, $P(z|T)$.

The following is an equality
\begin{equation}
P_D(z|T) = P(z|d\in D, T) = \int_{D} P(z_i|d_i, T) P(d_i|T) / P(d\in D|T) \; d d_i
\end{equation}
The factor $P(d|T)/P(d\in D|T)$ just serves to keep things normalized. It's the PDF for data $d$ for galaxies in $D$ given the ``true model'' $T$. In the limit of many galaxies $i=1,...,N_g$ that sample this PDF,
\begin{equation}
\int_{D} P(z_i|d_i,T) P(d_i|T) / P(d\in D|T) \; d d_i \approx \frac{1}{N_g} \sum_{i=1}^{N_g} P(z_i | d_i, T) \; ,
\end{equation}
and repeating this over all bins $D$ gives us a histogram approximation to $P(z|T)$.

This last equation shows that {\it} if we had access to the individual
galaxy posterior PDFs given the true model, then summing them over the
whole ensemble (each of which represents a sample dataset $d$ drawn from
$P(d|T)$) gives a good estimator of the underlying redshift
distribution. So, under what conditions does this ``stacked estimator''
provide a good approximation to the underlying $P(z|T)$? We can identify
two:
\begin{enumerate}
\item The galaxies in our dataset must have been drawn from $P(d|T)$, that is, they must be a representative subset of all the galaxies in the population we are trying to understand.
\item The model Universe that we assume when we compute the posterior PDF for each galaxy's redshift must be very close to the real Universe.
\end{enumerate}
As the discussion above shows, one way to achieve this second condition
is to work with a very large ``training set'' that permits us to make,
for the $i^{\rm th}$ galaxy whose redshift $z_i$ is to be inferred, a
subset  containing many galaxies that have observed data that is very
close to the target galaxy's data $d_i$. The distribution of redshifts
in this subset gives us an approximate sampling from $P(z_i|d_i,T)$.
This sort of scheme sounds a lot like machine learning, evoking visions
of methods like K-nearest neighbors.


\section{What if your model is not the true one?}

In the previous section, the noiseless observables (or latent variables) $a$ did not appear: we just made various engineering choices to enable us to make an approximation of $P(z_i|d_i,T)$ without ever considering what galaxies are like. If we do wish to understand galaxies, we need a generative model for them, $H$, which allows us to draw possible galaxy properties $a$ from $P(a|H)$. The hyper-parameters of this model, $h$, govern the shape of this PDF. Correctly guessing $H$ and inferring all the $h$'s is the endgame of galaxy evolution. In this section, we discuss two scenarios where an assumed model $H$ is employed, and the $h$ are unknown.


\subsection{Stacking despite model uncertainty}

What if we assume a model $H$ that is not the true model $T$? We can write down $P(a|h,H)$ and $P(h|H)$, and the individual galaxy posterior PDF becomes
\begin{align}
P(z_i | d_i, H) &= \int\int P(z_i|a_i) P(a_i|d_i,h,H) P(h|H) \; da_i\; dh \\
                &= \int\int P(z_i|a_i) P(a_i|h,H) P(h|H) P(d_i|a_i,h,H) / P(d_i|H) \; da_i \; dh \; .
\end{align}

Because $P(a,h)=P(a|h)P(h)$ and $P(d_i|a,h)$ depend on $h$, this PDF
will actually be broader than the distribution of $z$ that we would get
if we binned up the true redshifts of all galaxies that look like $d_i$.
Likewise, the stacked estimator made by summing posterior PDFs like
these will be broader than the actual PDF on an ensemble of galaxies
selected by $d\in D$. PDF stacking is not an operation that gets you an
accurate estimator of the underlying redshift distribution in a scheme
with uncertainty in the hyperparameters.

This means, for example, that stacking PDFs estimated with different
codes will give you the wrong PDF (both for a single galaxy and for an
ensemble) because that implicitly encodes uncertainty on the
hyperparameters (because different hyperparameters were used by the
different codes).

In a Bayesian hierarchical inference, all the parameters $a_i$ (for all
$i$) and hyperparameters $h$ are constrained by the data and integrated
out. But as long as they're not constrained completely, then stacking
the posterior predictive distributions $P(z_i|d_i,H)$ of individual
galaxies that some Hierarchical Bayesian gave you will not give an unbiased estimate of their true PDF -- it will
be wider than that. (Actually, that is already true for the PDF of an
individual galaxy. If you take spectra of many apparently identical
galaxies then the distribution of their redshifts will be narrower than
the above $P(z_i|d_i,H)$. This is because these spectra actually
constrain $h$ (in the limit for infinitely many spectra, they'll fully
constrain $h$ up to changes that are degenerate under the redshift
distribution of galaxies that look like $d_i$).


\subsection{Hierarchical inference}

Let's investigate the hierarchical inference itself, to look for clues
on how to combine posterior PDFs $P(z_i|d_i,M)$ given to us by people
working with an untrue model $M$. A Bayesian gives you an ensemble of
posterior PDFs---what do you do?

Let's go back to the simpler problem of inferring just the underlying
redshift distribution, $P(z|H)$, and suppose that we can evaluate the
likelihood $P(d|z,H)$ given our model. We recognize $P(z|h,H)$ as our
model redshift distribution, whose functional form is determined by $h$,
and that we are uncertain about the $h$. It is this $P(z|h,H)$ that we
need to propagate through into future analyses that require a redshift
distribution: the marginalization over $h$ needs to be postponed, so
that the degeneracy between the $h$ and any other hyperparameters (such
as the cosmological parameters) can be fully explored and captured.

Our goal, then, is to characterize the posterior PDF for $h$:
\begin{equation}
   P(h|d,H) = P(h|H) \int \prod_i P(d_i|z_i,H) P(z_i|h,H) / P(d_i|H) dz_i
\end{equation}
where we have assumed the independence of all our datasets. The
following derivation follows \citep{ForemanMackeyEtal2014}, culminating
in the same ``importance sampling trick.'' Further discussion of its
application to the photometric redshift problem can be found in Malz et
al (in preparation).\footnote{https://github.com/aimalz/prob-z}

The giant integral in the above equation is separable, and so we can write the integral of the product as a product of integrals:
\begin{equation}
   P(h|d,H) = P(h|H)/Z_H \prod_i \left[ \int P(d_i|z_i,H) P(z_i|h,H)  dz_i \right],
\end{equation}
where the normalization constant $Z_H = \prod_i P(d_i|H)$.
What our Bayesian collaborator provides us with is an ensemble of
posterior PDFs $P(z_i|d_i,M)$ obtained under the assumption of some
other model, $M$. If we insert each one into the numerator and
denominator of the appropriate integrand we get
\begin{equation}
   \int P(d_i|z_i,H) P(z_i|h,H)  dz_i = \int \frac{P(d_i|z_i,H) P(z_i|h,H)}{P(z_i|d_i,M)} P(z_i|d_i,M) dz_i
\end{equation}
If we can draw $N_s$ samples from the supplied posterior (or if it comes to us, approximately, in the form of $N_s$ samples), then this integral can be approximated as a sum:
\begin{equation}
   \int P(d_i|z_i,H) P(z_i|h,H)  dz_i \approx \frac{1}{N_s} \sum_{k=1}^{N_s} \frac{P(d_i|z_i^k,H) P(z_i^k|h,H)} {P(z_i^k|d_i,M)}
\end{equation}
This shows that each integral in our product over all galaxies can be
approximated by a sum of weights $w_i^k$, each of which is computed as a
combination of three PDFs evaluated at the sample redshift $z_k$: the
conditional prior in the hierarchical model $P(z_i^k|h,H)$, the
posterior PDF supplied to us $P(z_i^k|d_i,M)$, and the likelihood
$P(d_i|z_i)$:
\begin{equation}
   w_i^k = \frac{P(d_i|z_i^k) P(z_i^k|h,H)} {P(z_i^k|d_i,M)}.
\end{equation}
(Note that $P(d_i|z_i) = P(d_i|z_i,H) = P(d_i|z_i,M)$, since the data
only depend on the redshift, not the model which generated them.) Our
target ``prior'' PDF $P(z_i^k|h,H)$ appears in the numerator, showing
our route to the inference of $h$: we might consider sampling, or
optimizing, for $h$ via the
\begin{equation}
    \log P(h|d,H) = \log P(h|H) + \sum_{i=1}^{N_g} \log \sum_{k=1}^{N_s} w_i^k + {\rm const.}
\end{equation}
The scipy function \texttt{logsumexp} provides a numerically stable way
of summing the weights.

As consumers of photo-z code outputs, let's consider two cases we might
expect to encounter in practice:
\begin{enumerate}
    \item {\bf Tabulated posteriors.} In this case, we are given
    $P(z_i|d_i,M)$ in the form of a 1D tabulated function. We can
    interpolate this to compute the denominator of $w_k$, and we can
    draw samples from $P(z_i|d_i,M)$ straightforwardly. Evaluating
    $P(d_i|z_i^k)$ is also straightforward, {\it provided the data $d_i$
    really do only depend on the parameter $z_i$}.

    \item {\bf Sampled posteriors.} In this case, we don't have to draw
    samples ourselves, but on the other hand we do need a smooth
    approximation to the posterior $P(z_i|d_i,M)$ so we can evaluate it
    in the denominator of each weight -- and then we have the same task
    of computing the likelihood $P(d_i|z_i^k)$.
\end{enumerate}

In general, the photometric data will depend on the template (or galaxy
type) $t$ as well as the redshift, which means computing the
(marginalized) likelihood term is not easy. A better solution would be
to be given the value of the log prior [$\log P(z_i^k,t_i|M)$] along
with each sample's redshift and type. In this case, we can write
\begin{align}
   w_i^k &= \frac{P(d_i|z_i^k) P(z_i^k|h,H)} {P(z_i^k|d_i,M)}, \\
         &\propto \frac{P(z_i^k,t_i^k|h,H)} {P(z_i^k,t_i^k|M)}
\end{align}
Here, the likelihood term has cancelled out, and we are left with
weights that are simply ratios of the ``interim'' prior (applied by the
photometric redshift code) and the hierarchical prior. We divide out the
former and then apply the latter, in what can now be seen to be an
importance sampling approach. Note that to be able to compute this, we
need posterior samples in $t_i$ as well as $z_i$, and the log interim
prior values as well.



\section{Conclusions}

\begin{itemize}
    \item If you have a machine learning photometric redshift code that
    returns individual posterior PDFs $P(z_i | d_i, I)$ by interpolating
    in data space over a large and representative training set, then an
    approximation to the underlying redshift distribution $P(z | T)$ is
    given by the stacked estimator     $\frac{1}{N_g} \sum_{i=1}^{N_g}
    P(z_i | d_i, I)$.

    \item If you have a template-fitting photometric redshift code that
    returns individual posterior PDFs $P(z_i | d_i, M)$ by assuming
    model SEDs or types $t_i$, then the underlying redshift distribution
    can be modeled hierarchically as $P(z | h, H)$, and the
    hyperparameters $h$ be inferred using a double sum over galaxies and
    their weighted posterior samples,   $\log P(h|d,H) = \log P(h|H) +
    \sum_{i=1}^{N_g} \log \sum_{k=1}^{N_s} w_i^k + {\rm const,}$ where
    each weight is the ratio of the model redshift distribution and the
    ``interim'' prior assigned to the resdhift and type during the
    photometric code execution.

    \item The hierarchical inference above requires posterior photo-z
    samples to be accompanied by their log prior values.
\end{itemize}

\bibliographystyle{apj}
\bibliography{references}

\end{document}
